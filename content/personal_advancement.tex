\section{Personal Advancement}
\label{sec:personal_advancement}
\lhead{\thesection \space Personal Advancement}

As it was defined in the personal development report, I came up with a set of personal goals that I wanted to reach. To do so, I wanted to familiarise myself with various technologies in addition to the skills defined in \textit{\ref{sec:development_activities} \nameref{sec:development_activities}}.
\newline
Each section in this chapter details some topic that was relevant to the project work. It is described how I made use of the mentioned technology and how it has benefited me.

\subsection{\textit{React Native}}
\label{ssec:react_native}

The usage of \textit{React Native} was at the core of the project. The component-based nature of the framework allowed me to gain skills in the usage of interactive components and their interdependent structure. After working on the project, I am able to explain the core structure of \textit{React Native} and by extension that of \textit{React.js}. I am furthermore able to make use of \textit{React Native} to create OS-independent mobile applications from scratch as well as to work on existing \textit{React Native} applications, understanding their structure and behaviour.
\newline
After designing, implementing and modifying them myself, I am able to understand the definition of \textit{React Native} components. I am able to design a system that makes use of said components, including the definition as to which component needs to make use of any other component.

\subsection{\textit{Firebase} \& \textit{GraphQL}}
\label{ssec:firebase_graphql}

\begin{figure}[H]
    \centering
    \includegraphics[width=0.2\textwidth]{images/graphql-logo.png}
    \caption{\textit{GraphQL} Logo}
    \label{fig:graphql_logo}
\end{figure}

\textit{Firebase} and \textit{GraphQL} are both incorporated into the \textit{Connected.Football} architecture. \textit{GraphQL} is made use of to transfer information between the \textit{React Native} application and the backend server. As part of the implementation, I have applied \textit{GraphQL} mutations to send and retrieve information this way. After working on the project, I understand the concept of \textit{GraphQL} and am able to apply it in the \textit{JavaScript} language in combination with modules retrieved using \textit{NPM}. With a small amount of effort, I can also incorporate this knowledge into different programming languages and frameworks.

\begin{figure}[H]
    \centering
    \includegraphics[width=0.2\textwidth]{images/firebase-logo.png}
    \caption{\textit{Firebase} Logo}
    \label{fig:firebase_logo}
\end{figure}

While \textit{Firebase} was not a part of the \textit{Vote4Fun} extension that was eventually implemented, I researched the possibilities of using it for an on-boarding procedure. This procedure could be used to invite non-registered users to the application using their email addresses or phone numbers. \textit{Firebase} includes a complete pipeline to enable new users to register with ease. I am able to explain how this procedure works and I am able to apply it in an application that has a user management system already set up.

\subsection{Management using Atlassian \textit{Jira}}
\label{ssec:jira}

Even though it was not specifically related to any of the learning goals mentioned in \textit{\ref{sec:development_activities} \nameref{sec:development_activities}}, I wanted to familiarise myself with Atlassian \textit{Jira}, a project tracking software that was utilised over the course of the project. It was my goal to gain skills in how to use the software as well as to comprehend how the software can enhance agile software development frameworks such as \textit{Scrum}, by providing boards and other tracking method to keep track of sprints, tasks, issues and the development backlog.

\begin{figure}[H]
    \centering
    \includegraphics[width=0.3\textwidth]{images/jira-logo.png}
    \caption{Atlassian \textit{Jira} Logo}
    \label{fig:jira_logo}
\end{figure}

During the project, I used \textit{Jira} a lot. The tool was used to keep track of which team member was working on certain tasks as well as what tasks are still to be finished. \textit{Jira} also enabled me to store tasks that only became clear during development for future development planning.
\newline
After using the tool during the project, I can now apply usage of \textit{Jira} in future project and I am able to comprehend how \textit{Jira} was utilised during this project. I can point out how tasks were stored and managed over the course of the project and am able to locate tasks and planning efficiently, specifically those assigned to me.

\subsection{\textit{Node.js} Package Management}
\label{ssec:nodejs_package_management}

The application developed during the project makes heavily use of modules loaded using the \textit{Node.js} package manager (\textit{NPM}). These modules are open source and available for free on the internet.

\begin{figure}[H]
    \centering
    \includegraphics[width=0.3\textwidth]{images/node-js-logo.png}
    \caption{\textit{Node.js} Logo}
    \label{fig:node_js_logo}
\end{figure}

At the end of the project, I am able to use \textit{NPM} to load distinct modules so that I can make use of them in my own implementation. I am able to create \textit{Node.js} and \textit{React Native} applications from scratch as well as to comprehend which modules and packages are made use of in an existing application. Furthermore, I am able to remove packages, check and change their version as well as to check packages for deprecation.

\subsection{\textit{recompose} \& Functional Components}
\label{ssec:recompose}

As mentioned in \textit{\ref{ssec:software_implementation} \nameref{ssec:software_implementation}}, a framework that was used during the implementation was \textit{recompose}. The framework can be used to provide business logic to a functional \textit{React Native} component.
\newline
Over the course of the project I learned how to make use of \textit{recompose}. I applied the various functionality of the framework to enhance stateless components with state and logic. I am now able to understand the behaviour of components using \textit{recompose} as well as to develop my own components utilising it.
\newline
Since \textit{recompose} makes use of functional components, I also learned how to develop these functional components myself. I am now able to understand the differences between a functional, stateless component and a class component. I am able to develop both myself as well as to use both in the same context. I am also able to create functional component using \textit{recompose} using an existing class component as a base for development.

\subsection{Mobile end-to-end testing}
\label{ssec:mobile_e2e_testing}

As part of my research I have worked with various testing frameworks for \textit{React Native}. I tried to make use of all these frameworks to develop end-to-end tests for the project. The results of that endeavour can be seen in the research paper in the Appendix of the Group Dossier.
\newline
After researching the frameworks, I am now able to write end-to-end tests with the frameworks \textit{Appium}, \textit{Detox} and \textit{Cavy}, as well as testing individual components using \textit{Jest}. I can now understand how each framework interacts with a developed application and can make use of the framework to instruct it to simulate defined actions.